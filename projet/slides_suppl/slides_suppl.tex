\documentclass[pdf]{beamer}

\usepackage[utf8]{inputenc}
\usepackage[T1]{fontenc}
\usepackage[francais]{babel}
\usepackage{graphicx}
\usepackage{circuitikz}
\usepackage{lmodern}
\usepackage[squaren, Gray]{SIunits}
\usepackage{sistyle}
\usepackage[autolanguage]{numprint}
\usepackage{pgfplots}
\pgfplotsset{compat=1.9}
\usepackage{amsmath,amssymb,array}
\usepackage{tabularx}
% New command pour la modélisation mécanique, tri à effectuer
\newcommand\fv[1]{{\bf #1}} % free vector
\newcommand\fvd[1]{\dot{\bf #1}} % free vector derivated
\newcommand\fvdd[1]{\ddot{\bf #1}} % free vector derivated
\newcommand\fvr[1]{\mathring{\bf #1}} % free vector relatively derivated
\newcommand\fvrr[1]{\overset{\circ\circ}{\bf #1}} % free vector relatively derivated
\newcommand\uv[1]{{\bf\hat{ #1}}} % unit vector
\newcommand\ui{{\bf\hat{I}}} % unit vector I
\newcommand\uj{{\bf\hat{J}}} % unit vector J
\newcommand\uk{{\bf\hat{K}}} % unit vector K
\newcommand\wrt[2]{\ensuremath{\tensor*[_{ #1}]{ #2}{}}} % With Respect To
\newcommand\wtr[3]{\ensuremath{\tensor*[_{ #1}]{ #2}{^{ #3}}}} % With Two Respect
\newcommand\omegaf{{\bm \omega}}
\newcommand\omegafr{\mathring{\bm \omega}}
\newcommand\omegafd{\dot{\bm \omega}}
\newcommand\omegaft{\tilde{\bm \omega}}
\newcommand\omegaftr{\mathring{\tilde{\bm \omega}}}
\newcommand\omegat{\tilde{\omega}}
\newcommand\omegatd{\tilde{\dot{\omega}}}
\newcommand\ine{{\bf I}}
\newcommand\st{{\bf L}}
\newcommand\pst{{\bf M}}
\newcommand\lm{{\bf N}}
\newcommand\am{{\bf H}}
\newcommand\amd{\dot{\am}}
\newcommand\fo{{\bf F}}
\newcommand\po{\mathcal{P}}
\newcommand\xg{\ensuremath{\fv{R}}}
\newcommand\xgd{\ensuremath{\fvd{R}}}
\newcommand\xgdd{\ensuremath{\fvdd{R}}}
\newcommand\dvec[1]{\dot{\vec{ #1}}}
\newcommand\ddvec[1]{\ddot{\vec{ #1}}}
\newcommand\qp{\dot{q}}
\newcommand\dqp{\Delta \dot{q}}
\usepackage{url}
\usepackage{hyperref}
\hypersetup{
    colorlinks,
    citecolor=black,
    filecolor=black,
    linkcolor=black,
    urlcolor=black
}
\usetheme{Warsaw}
\mode<presentation>{}

\title{LFSABP1502 - Projet 2 : Concevoir un haut-parleur}
\author{Groupe 115.3}
\date{\today}

\begin{document}


\begin{frame}{Slides supplémentaires}
\begin{itemize}
\item  Lois physiques clefs;
\item  Filtre passe-bas;
\item  Filtre passe-haut;
\item  Filtre passe-bande;
\item  Dimensionnement électroaimant;
\item  Dimensionnement bobine mobile;
\item  Modélisation mécanique;
\item  Volume du caisson;
\item  Approximation fréquence de coupure;
\item  Planning;
\item  Cahier des Charges;
\item  Gain amplificateur
\item  Vue d'ensemble du circuit
\item  Bande passante
\end{itemize}
\end{frame}
%------------------------------------------------------
\begin{frame}{Lois physiques clefs}

Loi d'\textsc{Ampère} :
$$B = \mu_0\mu_r\frac{NI}{e}$$

Force de \textsc{Laplace} :
$$\vec{F} = i(t)\vec{L}\times{\vec{B}}$$ 

Loi de \textsc{Hooke} :
$$\vec{F} = -k \vec{x}$$ 

\end{frame}

%----------------------------------
\begin{frame}{Filtre passe-bas}

	\begin{figure}[ht]
\minipage{0.4\textwidth}
\begin{figure}[ht!]
    \centering
    \includegraphics[scale=0.3]{passe_bas.png}
\end{figure}
\endminipage\hfill
\minipage{0.35\textwidth}
\underline{Équation différentielle}
$$V_{in} = V_R + V_{out}$$
$$V \cdot \cos (\omega t) = RC\frac{dV_C}{dt} + V_C$$
$$RCy' + y = V \cdot \cos (\omega t)$$
\endminipage\hfill
\end{figure}
$V_{out} = \frac{V}{\sqrt{1 + R^2\omega^2C^2}}
\left(-\frac{e^{\frac{-t}{RC}}}{\sqrt{1 + R^2\omega^2C^2}} + \cos(\arctan(-RC\omega) + \omega t)\right)
$

\end{frame}

%--------------------------------------------------------------
\begin{frame}{Filtre passe-haut}

	\begin{figure}[ht]
\minipage{0.4\textwidth}
\begin{figure}[ht!]
    \centering
    \includegraphics[scale=0.3]{passe_haut.png}
\end{figure}
\endminipage\hfill
\minipage{0.35\textwidth}
\underline{Loi de Kirchhoff}
$$V_{in} = V_R + V_C$$
$$V_R = V_{in} - V_C$$
\endminipage\hfill
\end{figure}
$V_{out} = \frac{V}{\sqrt{1 + R^2\omega^2C^2}}
\left(\frac{e^{\frac{-t}{RC}}}{\sqrt{1 + R^2\omega^2C^2}} - \cos(\arctan(-RC\omega) + \omega t) \right)$
\centering{$+\cos(\omega t)$}

\end{frame}

%-----------------------------------------------------------
\begin{frame}{Filtre passe-bande}

$$V_{out,3} = \frac{V_{out,1} \cdot V_{out,2}}{V_{in,1}}$$

\begin{figure}[ht!]
\centering
\begin{tikzpicture}[>=stealth]
\begin{axis}[
xmin=0,xmax=2000,
ymin=0,ymax=1.2,
axis x line=middle,
axis y line=middle,
axis line style=->,
xlabel={$f$},
ylabel={$V_{out} / V_{in}$},
legend entries = {passe-bande, fréq. coupure 1, fréq. coupure 2}
]

\addplot[no marks,red,-] expression[domain=0:2000,samples=100]
{(((2*3.14*x*10000*0.00000047)*(1 + (2*3.14*x*10000*0.00000047)^2)^(-0.5))*
((1 + (2*3.14*x*500*0.00000047)^2)^(-0.5)))};
\addplot[no marks, gray] coordinates
{(33.86,0) (33.86,1.1)};
\addplot[no marks, gray] coordinates
{(677.26,0) (677.26,1.1)};
\end{axis}
\end{tikzpicture}
\caption{Graphe de $V_{out} / V_{in}$ pour le filtre passe-bande, pour les valeurs
suivantes: $R_{1} = \unit{500}{\ohm}$, $R_{2} = \unit{10000}{\ohm}$, et
$C = {\unit{0.00000047}{\farad}}$}
\label{pbf_ratio}
\end{figure}
\end{frame}
%--------------------------------------------------------
\begin{frame}{Dimensionnement électroaimant}
\underline{Champ magnétique dans l'entrefer}
$$H_e \cdot e = N_1 I \Rightarrow \frac{B_e}{\mu_0 \mu_r} e = N_1 I$$
$$B_e = \unit{0.07539825}{\tesla}$$
\underline{Résistance totale bobine}
$$L_{fil} = N_1 \cdot 2\pi r$$ 
$$L_{fil} = \unit{42.22}{\meter}$$
$$R = L_{fil} \cdot R_{lin} = 7.6 \Omega$$
\end{frame}

%-------------------------------------------------
\begin{frame}{Dimensionnement bobine mobile}
\underline{Force de \textsc{Laplace}}
$$\vec{F} = i(t)\vec{L}\times{\vec{B}}$$ 
\underline{Nombre de spires}
$$I = \frac{P}{V} \cdot{\sqrt{2}} = \unit{0.2357}{\ampere}$$
$$IL_{fil}B = kx \Rightarrow L_{fil} = \frac{kx}{IB} = \unit{9.57}{\meter}$$
$$L_{fil} = N_2 \cdot 2\pi r \Rightarrow N_2 = \frac{L_{fil}}{2\pi r} = 89.6$$
\underline{Résistance totale bobine}
$$R = L_{fil} \cdot R_{lin} = 1.72 \Omega$$
\end{frame}
%--------------------------------------------------
\begin{frame}{Modélisation mécanique}
\underline{Équation du mouvement}
$$m\fvdd{x}(t) = -k\fv{x}(t) + BLi(t)$$
$$m\fvdd{x}(t) + k\fv{x}(t) = \frac{B2\pi rNV_0}{R}\cos (\omega t)$$
\underline{Résolution de l'équation différentielle}
$$\fv{x}_h(t) = C\cos(\sqrt{\frac{k}{m}}t) - D\sin(\sqrt{\frac{k}{m}}t)$$
$$\fv{x}_p(t) = \frac{B2\pi rNV_0}{R(-m\omega^2 + k)} \cos (\omega t)$$
$$\fv{x}(t) = C\cos(\sqrt{\frac{k}{m}}t) - D\sin(\sqrt{\frac{k}{m}}t) + \frac{B2\pi rNV_0}{R(-m\omega^2 + k)} \cos (\omega t)$$
\end{frame}
%------------------------------------------------------
\begin{frame}{Volume du caisson}
\framesubtitle{Les paramètres de \textsc{Thiele} et \textsc{Small}}
$$\frac{F_b}{F_s} = \sqrt{\frac{V_{as}}{V_b} + 1}$$
$V_b$ volume du caisson \\
$F_s$ fréquence de résonance du haut-parleur à l'air libre \\
$F_b$ fréquence de résonance du haut-parleur fermé\\
$V_{as}$ volume d'air équivalent à la suspension du haut-parleur\\
\bigbreak
\underline{Si} $V_b\rightarrow \infty$ \\
\underline{alors} 
la fréquence de résonance égale celle à l'air libre.
\end{frame}
%--------------------------------------------
\begin{frame}{Approximation de la fréquence de coupure}
\framesubtitle{Filtre passe-bas}

\begin{figure}[ht]
\minipage{0.4\textwidth}
\begin{figure}
\centering
\begin{tikzpicture}[scale=0.5]
\begin{axis}[
xlabel={$f$},
ylabel={$V_{out}$},]
\addplot coordinates
{(0,2.5) (5000,2.5) (10000,2.5) (16000,1.7) (18000,1.55) (20000,1.45)};
\end{axis}
\end{tikzpicture}
\caption{Graphe de $V_{out}$ expérimentale en fonction de la fréquence}
\label{experimental_data}
\end{figure}
\endminipage\hfill
\minipage{0.4\textwidth}
\begin{figure}[!htb]
\centering
\begin{tikzpicture}[scale=0.5]
\begin{semilogxaxis}[ xlabel=Frequence, ylabel= $\frac{V_{out}}{V_{in}}$, xmin=1, xmax=1000000, ymin=0, ymax=1.2]

\addplot[domain=1:1e6, black] {(1/((1 + (2*3.14*x*57.5*0.00000047)^2)^(0.5))};
\addplot[no marks, red] coordinates
{(1,1) (5889,1)};
\addplot[no marks, red] coordinates
{(5889,1) (104811,0)};
\addplot[no marks, gray] coordinates
{(5889,0) (5889,1.2)};
\end{semilogxaxis}
\end{tikzpicture}
\caption{Confrontation du graphe théorique et des droites expérimentales}
\label{theory_vs_experiment}
\end{figure}
\endminipage\hfill
\end{figure}
\end{frame}
%---------------------------------------------
\begin{frame}{Approximation de la fréquence de coupure}
Système approché :
\begin{center}
\begin{array}{rcel}
$$
\begin{pmatrix}
4.204 & 1\\
4.255 & 1 \\
4.301 & 1
\end{pmatrix} &

\begin{pmatrix}
a\\
b
\end{pmatrix} &

= &

\begin{pmatrix}
1.607\\
1.565\\
1.524
\end{pmatrix}
$$
\end{array}
\end{center}

$$\fbox{y = -1.96 \cdot \log(x) + 9.84}$$

$$\fbox{x = 5557.7 Hz}$$

Fréquence théorique : $f = \frac{1}{2\pi RC}$ = 5889.2 Hz
\newline

Avec R = 57.5$\Omega$ et C = 470 nF
\end{frame}
%--------------------------------------------
\begin{frame} {Planning}
\begin{figure}[ht!]
    \centering
    \includegraphics[scale=0.5]{planning.png}
\end{figure}
\end{frame}
%----------------------------------------
\begin{frame} {Cahier des charges}
\begin{figure}[ht!]
    \centering
    \includegraphics[scale=0.45]{cdc.png}
\end{figure}
\end{frame}

\begin{frame}{Gain de l'amplificateur}
\begin{figure}[ht]
\minipage{0.4\textwidth}
\begin{figure}[ht!]
    \centering
    \includegraphics[scale=0.4]{ampli.png}
\end{figure}
\endminipage\hfill
\minipage{0.4\textwidth}
$$V_{in} = \frac{R_1}{R_1 + R_2} V_{out}$$
$$A = \frac{V_{out}}{V_{in}} = \frac{R_1 + R_2}{R_1}$$
\bigbreak
Pour réduire le gain $A$ à $1$, deux possibilités:
\begin{itemize}
\item	Choisir $R_1 >> R_2$ ;
\item Choisir $R_2 = 0$ ;
\end{itemize}
\endminipage\hfill
\end{figure}
\end{frame}

%-----------------------------------------

\begin{frame}{Vue d'ensemble du circuit}
\begin{figure}[ht!]
    \centering
    \includegraphics[scale=0.4]{plaquette.png}
\end{figure}
\end{frame}

%------------------------------------
\begin{frame}{Bande passante}
\begin{figure}[ht]
\minipage{0.4\textwidth}
\begin{figure}[ht!]
\centering
\begin{tikzpicture}[scale=0.5]
\begin{axis}[
xmin=0,xmax=2000,
ymin=0,ymax=1.2,
axis x line=middle,
axis y line=middle,
axis line style=->,
xlabel={$f$},
ylabel={$V_{out} / V_{in}$},
legend entries = {passe-bande, fréq. coupure 1, fréq. coupure 2}
]

\addplot[no marks,red,-] expression[domain=0:2000,samples=100]
{(((2*3.14*x*10000*0.00000047)*(1 + (2*3.14*x*10000*0.00000047)^2)^(-0.5))*
((1 + (2*3.14*x*500*0.00000047)^2)^(-0.5)))};
\addplot[no marks, gray] coordinates
{(33.86,0) (33.86,1.1)};
\addplot[no marks, gray] coordinates
{(677.26,0) (677.26,1.1)};
\end{axis}
\end{tikzpicture}
\caption{Graphe du gain d'un filtre passe-bande}
\label{pbf_ratio}
\end{figure}
\endminipage\hfill
\minipage{0.4\textwidth}
Pour avoir la plus grande bande passante, il faut donc :
$$f_1 < f_2$$
$$ \Rightarrow \frac{1}{2\pi R_1C} < \frac{1}{2\pi R_2C} $$
$$\Rightarrow R_2 < R_1$$
\endminipage\hfill
\end{figure}
\end{frame}



\end{document}
